% !TeX spellcheck = en_US

\chapter{Evaluation}
\label{chap:ch5}
For evaluating the usefulness of the prototype an expert review and survey is performed.
To acquire the questions for this survey a process based on the \gls{GQM} approach introduced by \cite{caldiera1994goal} is used.
The results from that are shown in \cref{sec:ch5:s1}.
\Cref{sec:ch5:s2} describes the design and realization as well as the results of the expert review and survey.
In \cref{sec:ch5:s3} the results of the survey as well as the validity of the thesis are discussed.
Finally, the identified threats to this validity are described in \cref{sec:ch5:s4}.

\section{GQM}
\label{sec:ch5:s1}
The process described in this section does not exactly follow the \gls{GQM} approach introduced by \cite{caldiera1994goal},
but is based on it.
The objective of the process is to find good questions, which can be used to evaluate the success of this thesis.
Based on these questions the expert survey can then be performed.

The first step, is to identify the goals of the work.
For this thesis the following could be identified as the primary and only goal \textbf{(G1)}:
Improve the efficiency, effectiveness, and convenience for developers when working with issues in the Gropius environment during their software development process. 

Next, questions have to be found, which can be used to evaluate, whether that goal is reached.
For that purpose the following three questions could be identified:
\begin{itemize}
	\item[\textbf{Q1}] What problems do developers face when working with issues affecting multiple independent projects during the software development/engineering process?
	\item[\textbf{Q2}] Which of these problems does this thesis's tool solve and to which degree?
	\item[\textbf{Q3}] Can the industry imagine using such a tool in production?
\end{itemize}

As it was clear from the start, that an expert review and survey would be conducted, no metrics need to be found for these questions.
However, it must be investigated if an expert review is a reasonable metric for these questions.
For \textbf{Q1} an expert survey is a reasonable metric, if the experts are developers, who currently work are have worked in a field with multiple independent projects or have enough experience to put them self into that situation.
It is an adequate metric for \textbf{Q2}, if the experts fulfill the conditions above and have used the thesis' tool or at least were given an introduction about the features of the tool.
An expert survey is a reasonable metric for \textbf{Q3}, if the experts are or have been working in the industry and fulfill the above conditions.

Therefore, the evaluation metric (\textbf{M1}) for all questions is an expert review and survey with the correct experts, who are introduced to \gls{GropiusEI}. 

\section{Expert Survey}
\label{sec:ch5:s2}
The main element of this evaluation is the expert survey, where selected experts answer a set of prepared questions. 
But first an expert review of \gls{GropiusEI} is performed.
This makes sure the experts know the tool and its features and has the additional benefit of direct feedback about the prototype.
The process of the review and survey is explained in \cref{ssec:ch5:ss2.1}.
The results from both are presented in \cref{ssec:ch5:ss2.2}.

\subsection{Design and Realization}
\label{ssec:ch5:ss2.1}
To perform an expert review and survey, first a group of experts has to be found.
To achieve this various developers from academia and industry were contacted and asked to participate in this review.
If they were generally interested, the thesis' concept and the procedure of the review and survey were explained to them.
In total 21 experts where contacted, of which 12 agreed to take part.

As preparation for the expert review, a detailed guide on installing \gls{GropiusEI} was produced
to ensure experts could install it on their own device if they wanted.
It can be found in \cref{sec:appendix:er:install_guide} and
the exact release used for the review is attached as \cref{sec:appendix:er:release}.

Additionally, a scenario was created, in which experts could test the tool in as real as possible conditions.
This scenario consists of an eclipse workspace with a java-based demo software using multiple components \footnote{\url{https://github.com/eventuate-tram/eventuate-tram-examples-customers-and-orders}} and a \gls{GropiusEI} data file.
This data file contains the correct components and interfaces for the workspace as well as some developers and labels in addition to
a number of invented issues.
These issues are as realistic as possible with a title that could be found in a live \gls{IMS} somewhere, a feasible text body,
typical labels, sometimes linked issues, and most with at least one location in the workspace.
Furthermore, a few tasks were prepared, which were used to introduce each feature to the expert.
These tasks were typical things a developer would need to do during his workday like finding an issue in the code are creating a new issue.
For each task a detailed explanation, how the task can be done using \gls{GropiusEI} is also included. 
The scenario can be found in \cref{sec:appendix:er:scenario} and a short explanation as well as the mentioned tasks in \cref{sec:appendix:er:scenario_explanation}.

After an expert agreed to participate they were provided with three options for executing the expert review and asked which they preferred.
The first option was to install the tool and the scenario on their own device with the help of the mentioned installation guide.
Then an online meeting would be scheduled, such that the tasks from the scenario could be completed while I watch with the help of a screen sharing solution. That way I could help the experts with any problems and directly answer questions as well as guide them through the prepared tasks instead of sending the tasks.
The second option was, that the expert installs both on their own device and go through the scenario doing the tasks on their own.
In this case, they were sent the tasks including the explanations in a written document.
The last option was, that I would present the tool in an online meeting using screen sharing tools. That way the expert did not need to install anything on their device. For that reason, this was also the fastest option. On the other hand, the expert could not use the tool himself, but just watch me do the tasks.
In every case, the expert was asked for general feedback.

After an expert completed the review he was sent the following questions and asked to answer them:
\begin{enumerate}
	\item What do you see as the most significant problems when working with issues affecting multiple independent projects during your software development/engineering process? \label{itm:ch5:expert_questions_problems}
	\item Do you think the following points represent problems or challenges of existing issue management systems you face as a developer? \label{itm:ch5:expert_questions_are_these_problems}
	\subitem The lack or insufficient support of linking issues to multiple specific software development artefacts accurate to the line, especially if the artifacts are part of multiple independent projects.
	\subitem The lack or insufficient support of linking issues to each other
	\subitem The lack or inadequate support of navigating from one to the next issue through the chain of related or dependent issues.
	\item For which of these problems (from points 1 and 2) would you say Gropius EI completely solves them? \label{itm:ch5:expert_questions_solve}
	\item For which of these problems (from points 1 and 2) would you say Gropius EI reduces them? \label{itm:ch5:expert_questions_reduce}
	\subitem How far would you say are each of these problems reduced?
	\item Could you imagine to use Gropius EI in your day-to-day business?
\end{enumerate}
These questions are based on the evaluation questions \textbf{Q1} to \textbf{Q3} presented in \cref{sec:ch5:s1}.
Question \ref{itm:ch5:expert_questions_are_these_problems} was added in order to get good answers to questions \ref{itm:ch5:expert_questions_solve} and \ref{itm:ch5:expert_questions_reduce} even if the expert did not find a lot of answers to question \ref{itm:ch5:expert_questions_problems}.
The points for question \ref{itm:ch5:expert_questions_are_these_problems} were conceived in an internal brainstorming session.

\subsection{Results}
\label{ssec:ch5:ss2.2}
%Of the 12 experts, who agreed to participate, a total of $x$ reviewed the tool and $y$ answered the prepared questions in the end. \todo{Insert correct numbers}
\section{Discussion}
\label{sec:ch5:s3}
\section{Threats to Validity}
\label{sec:ch5:s4}

\begin{figure}[!h]
	\centering
	\tikzset{
		my node style/.style={
			font=\small,
			rectangle,
			rounded corners,
			minimum size=6mm,
			drop shadow,
			draw=blue!40, 
			fill=blue!20, 
			very thick,
			align=center,
		}
	}
	\forestset{
		my tree style/.style={
			for tree={
				my node style,
				parent anchor=south,
				child anchor=north,
				l sep+=5pt,
				edge={draw=blue!40, very thick},
				edge path={
					\noexpand\path [draw, \forestoption{edge}] (!u.parent anchor) -- +(0,-7.5pt) -| (.child anchor)\forestoption{edge label};
				},
			    delay={if content={}{shape=coordinate}{}}
			}
		}
	}
	\begin{forest}
	my tree style
	[Threats to Validity
      [External Validity
	    [\begin{varwidth}{0.16\linewidth}Small sample size\end{varwidth}]
	    [\begin{varwidth}{0.16\linewidth}Sample not representative\end{varwidth}]
	  ]
	  [Construct Validity 
	    [[[
	      [\begin{varwidth}{0.16\linewidth}Questions for experts misrepresent research questions\end{varwidth}]
	      [\begin{varwidth}{0.16\linewidth}Questions misunderstood by experts\end{varwidth}]
	      [\begin{varwidth}{0.16\linewidth}Answers misinterpreted\end{varwidth}]
	      [\begin{varwidth}{0.16\linewidth}Only one kind of measurement\end{varwidth}]
	      [\begin{varwidth}{0.16\linewidth}Expert survey is not objective nor statistical\end{varwidth}]
	    ]]]
	  ]
	  [Reliability
	      [\begin{varwidth}{0.16\linewidth}Documents written in German\end{varwidth}]
	      [\begin{varwidth}{0.16\linewidth}Artifacts hosted online may not be available forever\end{varwidth}]
	  ]
	  [Internal Validity]
	]
	\end{forest}
  
	\caption{Overview over the Threats to Validity}
	\label{fig:threatsToValididty}
\end{figure}

This section discusses which threats to the validity of the results discussed above could be identified.
An overview of these threats can be seen in ...
The threats can be grouped into four categories as stated by \cite{runeson2009guidelines}.

The first category investigated, is the internal validity.
This aspect of validity is concerned with whether the observed results are actually caused by the changes introduced by this work,
or if the results could be effects of any unknown influence \cite{runeson2009guidelines}.
But as the experts were directly asked if \gls{GropiusEI} solves or reduces the stated problems, no threat to this aspect of validity could be identified.

The next kind is the construct validity.
It represents the correctness of the assumption, that the collected results actually correspond to the research questions \cite{runeson2009guidelines}.
For this aspect a few threats could be identified.
First, the questions sent to the experts, which are presented in \cref{ssec:ch5:ss2.1} could misrepresent the research questions 
introduced in \cref{sec:ch5:s1}.
Next, these questions could have been misunderstood by one or more experts.
Furthermore, the answers of the experts could have been misinterpreted.
Moreover, just a single kind of measurements was performed.
Finally, an expert survey does not produce objective ore statistical results.
The results are subjective answers from the experts given in free text form, such that statistical evaluation is difficult. 

Another aspect is the external validity.
That is concerned with whether the results can be generalized beyond the small studied sample \cite{runeson2009guidelines}.
The following threats were identified for this category:
First, the sample size was very small compared to all developers working with issues in a relevant field.
Second, the experts were not picked randomly, but all had prior contact to me or my supervisor.
Therefore, the sample was not very representative of the complete population.

The last aspect of validity is the reproducibility or reliability.
It deals with any reasons a separate evaluation by other researchers reproducing this evaluation would not have the same conditions \cite{runeson2009guidelines}.
For this aspect no major threats could be identified, as the whole process is described in detail in \cref{ssec:ch5:ss2.1} and all artifacts used can be found in \cref{chap:appendix:expert_review_docs}.
The first minor threat found is the fact that these documents are written in German and would need to be translated for researchers or experts who don't speak German.
The other minor threat is that the artifacts hosted on the internet referenced in the appendix might not be available forever.