% !TeX spellcheck = en_US

\chapter{Conclusion}
\label{chap:ch6}
This chapter concludes this thesis by first giving a short summary in \cref{sec:ch6:s1}, highlighting the most important lessons learned during the process of this thesis in \cref{sec:ch6:s4}, and finally discussing what work could be based on this thesis in the future in \cref{sec:ch6:s5}.

\section{Summary}
\label{sec:ch6:s1}
The goal of this thesis was to develop, implement, and evaluate a concept for integrating \gls{Gropius} into the developer's \gls{IDE} and therefore improve the efficiency, effectiveness, and convenience for developers when working with issues in environments with many components and multiple teams during their software development process.

First, a concept for such an \gls{IDE} extension was developed based on a requirements engineering process.
This concept is independent of the \gls{IDE} and can therefore be used to develop an extension for any \gls{IDE}.
At the core of the concept is the issue list, and the issue detail \gls{UI} element.

Then, an \gls{Eclipse} extension has been implemented based on this concept.
Large parts of it are created using a \gls{MDSD} approach. Therefore a lot of the code is generated.
Even though various features could not be implemented in time, the result helps demonstrate the concept.
It can easily be developed into a full issue management extension using \gls{Gropius}.

Using this extension, an expert review has been performed, followed by an expert survey.
The former was used to gain feedback about the extension while the latter served to evaluate the concept.
While the experts generally like the concept, not all see the problems it is trying to solve.
However, most experts do agree with at least some of the problems and think \gls{GropiusEI} does solve them.
Finally, all but one expert can imagine using it in the future.
Based on the discussion of these results, it can be said that the thesis' goal is reached.

In the end, this thesis provides a detailed concept for issue management plugins working with the \gls{Gropius} framework as well as a basic implementation of it together with a lot of feedback for future improvements.

\section{Lessons Learned}
\label{sec:ch6:s4}
The most important thing I learned during the process of this thesis is using a \gls{MDSD} approach, especially working with \gls{EMF} and particularly \gls{Parsley}.
Both do not have a lot of documentation, but by working with them and looking into their source, I learned a lot about their use but also their internals.

Additionally, I gained practical experience with planning and performing an expert review and survey.
Especially creating the research questions using the \gls{GQM} approach and evaluating the threats to validity gave me a lot of insight into those areas.

Moreover, I also practiced and improved my literature research as well as scientific writing skills.
I especially learned searching in an area with very little scientific work and finding hidden papers.

\section{Future Work}
\label{sec:ch6:s5}
One possible first step for any future work is obviously improving \gls{GropiusEI}.
The remaining features proposed in the concept, as well as the ones suggested by the experts, could be implemented.
Once the extension is in a more usable form and the \gls{Gropius} back-end also has support for some \gls{IMS},
a study with many more developers could be performed in one or more companies.
An instance of the back-end could be set up within each company and linked to their existing \glspl{IMS} and other \gls{IMS} of projects they are using.
Then, the developers of the company could try using \gls{GropiusEI} in their daily business.
This study could reveal the performance increase gained by using the extension.

Alternatively, extensions for other \glspl{IDE} could be developed based on the concept of this thesis.
This helps to evaluate how easily it can be applied to those other \glspl{IDE} and how much work needs to be done to port the existing \gls{IDE}-independent parts of the extension to another extension.
It would need to be evaluated if it is easier to get the code generated by \gls{EMF} working with another \gls{IDE} or if it is easier to generate new code from the existing models with the help of another \gls{MDSD} framework. 