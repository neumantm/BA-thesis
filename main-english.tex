% !TeX document-id = {54d2d254-7585-48d2-9f78-dac02ef9b770}
% !TeX spellcheck = en-US
% !TeX encoding = utf8
% !TeX program = lualatex
% !BIB program = biber
% -*- coding:utf-8 mod:LaTeX -*-


% vv  scroll down to line 200 for content  vv

\PassOptionsToPackage{language=english}{uni-stuttgart-cs-cover}

\let\ifdeutsch\iffalse
\let\ifenglisch\iftrue
\input{pre-documentclass}
\documentclass[
  % fontsize=11pt is the standard
  a4paper,  % Standard format - only KOMAScript uses paper=a4 - https://tex.stackexchange.com/a/61044/9075
  twoside,  % we are optimizing for both screen and two-side printing. So the page numbers will jump, but the content is configured to stay in the middle (by using the geometry package)
  bibliography=totoc,
  %               idxtotoc,   %Index ins Inhaltsverzeichnis
  %               liststotoc, %List of X ins Inhaltsverzeichnis, mit liststotocnumbered werden die Abbildungsverzeichnisse nummeriert
  headsepline,
  cleardoublepage=empty,
  parskip=half,
  %               draft    % um zu sehen, wo noch nachgebessert werden muss - wichtig, da Bindungskorrektur mit drin
  draft=false
]{scrbook}
\input{config}


\usepackage[
  title={IDE Support of Issue Management for Multi-Component Architectures},
  author={Tim Neumann},
  type=bachelor,
  institute=istesqa, % or other institute names - or just a plain string using {Demo\\Demo...}
  course={Softwaretechnik},
  examiner={Prof. Dr.-Ing. Steffen Becker},
  supervisor={Dr. rer. nat. Uwe Breitenbücher,\\ Sandro Speth, M.Sc.},
  startdate={March 18, 2020},
  enddate={Novemver 17, 2020}
]{uni-stuttgart-cs-cover}

% Hier stehen alle Abkürzungen
\newacronym{IDE}{IDE}{integrated development environment}
\newacronym{UI}{UI}{user interface}
\newacronym{API}{API}{Application Programming Interface}
\newacronym{REST}{REST}{Representational state transfer}
\newacronym[plural={IMS}, \glsshortpluralkey={IMS}]{IMS}{IMS}{issue management system}
\newacronym{PNG}{PNG}{Portable Network Graphics}
\newacronym{UML}{UML}{Unified Modeling Language}
\newacronym{EMF}{EMF}{Eclipse Modeling Framework}
\newacronym{GropiusEI}{Gropius EI}{Gropius Eclipse Integration}
\newacronym{GQM}{GQM}{Goal Question Metric}

%\newacronym[plural={RDBMS},shortplural={RDBMS}]{rdbms}{RDBMS}{Relational Database Management System}


\makeindex

\begin{document}

%tex4ht-Konvertierung verschönern
\iftex4ht
  % tell tex4ht to create picures also for formulas starting with '$'
  % WARNING: a tex4ht run now takes forever!
  \Configure{$}{\PicMath}{\EndPicMath}{}
  %$ % <- syntax highlighting fix for emacs
  \Css{body {text-align:justify;}}

  %conversion of .pdf to .png
  \Configure{graphics*}
  {pdf}
  {\Needs{"convert \csname Gin@base\endcsname.pdf
      \csname Gin@base\endcsname.png"}%
    \Picture[pict]{\csname Gin@base\endcsname.png}%
  }
\fi

%\VerbatimFootnotes %verbatim text in Fußnoten erlauben. Geht normalerweise nicht.

\input{commands}
\pagenumbering{roman}
\Titelblatt

%Eigener Seitenstil fuer die Kurzfassung und das Inhaltsverzeichnis
\deftripstyle{preamble}{}{}{}{}{}{\pagemark}
%Doku zu deftripstyle: scrguide.pdf
\pagestyle{preamble}
\renewcommand*{\chapterpagestyle}{preamble}



%Kurzfassung / abstract
%auch im Stil vom Inhaltsverzeichnis
%\ifdeutsch
 % \section*{Kurzfassung}
%\else
 % \section*{Abstract}
%\fi

%<Short summary of the thesis>

%\cleardoublepage

\section*{Abstract}
\lipsum[1-1]
\cleardoublepage

\section*{Kurzfassung}
\lipsum[2-2]
\cleardoublepage
% BEGIN: Verzeichnisse

\iftex4ht
\else
  \microtypesetup{protrusion=false}
\fi

%%%
% Literaturverzeichnis ins TOC mit aufnehmen, aber nur wenn nichts anderes mehr hilft!
% \addcontentsline{toc}{chapter}{Literaturverzeichnis}
%
% oder zB
%\addcontentsline{toc}{section}{Abkürzungsverzeichnis}
%
%%%

%Produce table of contents
%
%In case you have trouble with headings reaching into the page numbers, enable the following three lines.
%Hint by http://golatex.de/inhaltsverzeichnis-schreibt-ueber-rand-t3106.html
%
%\makeatletter
%\renewcommand{\@pnumwidth}{2em}
%\makeatother
%
\tableofcontents

% Bei einem ungünstigen Seitenumbruch im Inhaltsverzeichnis, kann dieser mit
% \addtocontents{toc}{\protect\newpage}
% an der passenden Stelle im Fließtext erzwungen werden.

\listoffigures
\listoftables

%Wird nur bei Verwendung von der lstlisting-Umgebung mit dem "caption"-Parameter benoetigt
%\lstlistoflistings 
%ansonsten:
\ifdeutsch
  \listof{Listing}{Verzeichnis der Listings}
\else
  \listof{Listing}{List of Listings}
\fi

%mittels \newfloat wurde die Algorithmus-Gleitumgebung definiert.
%Mit folgendem Befehl werden alle floats dieses Typs ausgegeben
\ifdeutsch
  \listof{Algorithmus}{Verzeichnis der Algorithmen}
\else
  \listof{Algorithmus}{List of Algorithms}
\fi
%\listofalgorithms %Ist nur für Algorithmen, die mittels \begin{algorithm} umschlossen werden, nötig

% Abkürzungsverzeichnis
\printnoidxglossaries

\iftex4ht
\else
  %Optischen Randausgleich und Grauwertkorrektur wieder aktivieren
  \microtypesetup{protrusion=true}
\fi

% END: Verzeichnisse

\mainmatter
\pagenumbering{arabic}

% Headline and footline
\renewcommand*{\chapterpagestyle}{scrplain}
\pagestyle{scrheadings}
\pagestyle{scrheadings}
\ihead[]{}
\chead[]{}
\ohead[]{\headmark}
\cfoot[]{}
\ofoot[\usekomafont{pagenumber}\thepage]{\usekomafont{pagenumber}\thepage}
\ifoot[]{}


%% vv  scroll down for content  vv %%




























\todo{Check includes for todo comments}


%%%%%%%%%%%%%%%%%%%%%%%%%%%%%%%%%%%%%%%%%%%%%%%%%%%%%%%%%%%%%%%%%%%%%%%%%%%%%%
%
% Main content starts here
%
%%%%%%%%%%%%%%%%%%%%%%%%%%%%%%%%%%%%%%%%%%%%%%%%%%%%%%%%%%%%%%%%%%%%%%%%%%%%%%


% !TeX spellcheck = en_US

\chapter{Introduction}
New software today is often based on a microservice architecture.
Such software consists of many small independent services, all fulfilling a small portion of the complete task.
Often these services are implemented in many software projects by multiple teams.
As with any software development process it is advisable to have some kind of issue management solution as well as a way to document any major design decisions.
Furthermore, documenting the interfaces between the services may even be more important than documenting the interfaces between components of a typical application.

However as the documentation and issues are spread over multiple projects and each may affect more than one project there are several challenges to managing these documents and issues effectively.
For example changes to documentation may not be propagated to all teams. Another example is, that issues found by one team may have the same root cause as issues found by another team, however, the root cause may even be caused by a piece of code of a third team.

One possible approach to deal with this problem is introduced in \cite{Speth2019}.
The proposed solution are so-called multi-project coding issues. These are issues concerning multiple projects and, therefore, are stored in the issue management system of each relevant project. Additionally, traceability links between multiple issues and artifacts are proposed. As a concrete syntax a graphical notation for these issues and links is introduced and implemented in a prototype framework. The frontend of this framework is a web-based application with an architecture graph at its core.
This graph is used to view and edit the system architecture and issues displayed with the introduced notation. 
The system is a supportive tool for software architects and project managers to keep an overview over the services and dependencies between those and as well as the current issues and their influence on all components. 

Even though this tool can be used by developers, it is rarely necessary for a developer to look at the whole system. It is more important that developers understand what parts of the system depend on the piece of source code they are working on and which other components may influence the behavior of it. Similarly, when working on an issue a developer does not need to look at all issues of the complete system, but only at those, which may be a cause or an effect of the one being worked on. Finally, it should be as easy as possible for developers to create a new issue, think about and discuss how it may relate to other issues, and write down their findings. A web-app,  which needs to be opened, the correct service selected, and the relevant file searched are all obstacles, which may impede a developer during this process. 

To disrupt the workflow of a developer as little as possible and remove most of these obstacles, it would be beneficial to integrate this approach into the developer's \ac{IDE}. There he could be shown all issues attached to a file or service as well as issues in components depending on it. Furthermore, he could be enabled to create issues for the code he's currently looking at and he could be aided in referencing any other relevant issues. 

\todo{"This is not the proposal anymore"}
Therefore, this thesis following paper proposes a bachelor's thesis investigating development of an \ac{IDE} plugin for the successor of the  platform presented in \cite{Speth2019}, which is currently being developed by Speth. For this, a prototype Eclipse plugin integrating with this framework is planned.

This proposal is structured as follows: First, \autoref{ch:foundation} will introduce the foundations and related work for the thesis. Then, \autoref{ch:concept} will present the concept of what will be done, including the general approach and what will be implemented. Finally, \autoref{ch:organization} will discuss some organizational details such as scheduling and risk management.

\section*{Thesis Structure}
%Die Arbeit ist in folgender Weise gegliedert:
%\begin{description}
%\item[Kapitel~\ref{chap:ch2} -- \nameref{chap:ch2}:] Hier werden werden die Grundlagen dieser Arbeit beschrieben.
%\item[Kapitel~\ref{chap:conclusion} -- \nameref{chap:conclusion}] fasst die Ergebnisse der Arbeit zusammen und stellt Anknüpfungspunkte vor.
%\end{description}

% !TeX spellcheck = en_US

\chapter{Foundations and Related Work}
\label{chap:ch2}
This chapter outlines other work relevant to this thesis.
First, the foundations are described in \cref{sec:ch2:s1}.
Then \cref{sec:ch2:s2} introduces and discusses other work in this research area, 
found with the methods described in \cref{ssec:ch2:ss2.1}.

\section{Foundations}
\label{sec:ch2:s1}
This section describes the foundations, this work is based on and which may be useful for understanding this thesis.
First, \cref{ssec:ch2:ss1.1} covers the idea of software issues and issue management systems in general.
Then, a concept for issue management in environments where multiple software components, 
which are developed by independent teams, is described in \cref{ssec:ch2:ss1.2}.
\Cref{ssec:ch2:ss1.3} introduces the concept of \glspl{IDE} and how they can be extended 
with a focus on \gls{Eclipse} as the prototype developed for this thesis is an \gls{Eclipse} plugin.
Finally, \cref{ssec:ch2:ss1.4} covers two frameworks used in the implementation of that plugin.

\subsection{Software Issues and \acrlongpl{IMS}}
\label{ssec:ch2:ss1.1}
Software issues are a way to communicate a need for action in the software engineering process.
Traditionally, a distinction was made between a bug, which needs to be debugged or fixed, a task someone needs to do, 
and an enhancement being requested.
Issues can represent any of these things as well as anything a developer needs them to represent \cite{Atlassian2020Issue,Github2020Issues}.

\glspl{IMS}, also called issue tracking systems, are programs, which maintain these issues 
and ``help organizations manage issue reporting, assignment, tracking, resolution, and archiving'' \cite{bertram2010communication}.
Other common names include bug tracker, defect tracker, and ticket system.
According to \cite{janak2009issue} they can bring various benefits like improved software quality, request accountability, and 
an increase in productivity. 

Software issues typically consist at least of a title, a detailed description or body text, and a state, 
which typically has at least the possible values open and closed.
Additionally, each \gls{IMS} supports various additional fields like the type(bug, enhancement, task, etc.), 
labels to further classify the issue, relevant parts of the source code, the creation date, a due date, the issue creator, the developer(s) assigned to work on it, or many more.
However, the amount of supported fields varies greatly between different programs.

Common \glspl{IMS} include Atlassian's Jira \footnote{\url{https://www.atlassian.com/software/jira}}, 
Bugzilla \footnote{\url{https://www.bugzilla.org/}} 
as well as the issue management support integrated into Redmine \footnote{\url{https://www.redmine.org/}}, 
GitHub \footnote{\url{https://github.com/}} and 
GitLab \footnote{\url{https://gitlab.com/}}.
Most provide various features in addition to just storing these issues. 
Those features include linking related issues, grouping issues into so-called Milestones, organizing issues on project boards, similar to the Kanban Boards described in \cite{epping2011kanban}, advanced time tracking, and more.
An in-depth comparison of a few well known \glspl{IMS} is provided by \cite{janak2009issue}. 

\subsection{Issue Management in Multi-Team, Multi-Component Environments}
\label{ssec:ch2:ss1.2}
Managing documentation and issues for multi-team, multi-project scenarios efficiently is a major problem for the software development process  \cite{mahmood2015identifying}. This assertion is supported by some posts in forums of Jira and Redmine asking for solutions to this problem \footnote{\url{https://www.redmine.org/boards/1/topics/21939}}\footnote{\url{https://community.atlassian.com/t5/Jira-questions/How-do-others-work-with-issues-affecting-multiple-projects/qaq-p/399950}}\footnote{\url{https://community.atlassian.com/t5/Jira-Software-questions/Share-one-issue-quot-ticket-quot-across-multiple-projects-and/qaq-p/407534}}. Another strong argument supporting it are the results of an expert survey conducted for \cite{Speth2019}, which indicate that industry experts also see this as a problem.

One possible approach to deal with this problem is introduced in \cite{Speth2019}.
The proposed solution are so-called multi-project coding issues. These are issues concerning multiple projects and, therefore, are stored in the issue management system of each relevant project. Additionally, traceability links between multiple issues and artifacts are proposed. As a concrete syntax, a graphical notation for these issues and links is introduced and implemented in a prototype framework. The frontend of this framework is a web-based application with an architecture graph at its core.
This graph is used to view and edit the system architecture and issues displayed with the introduced notation. 

\cite{speth2020gropius} further refines this concept and introduces a tool named \gls{Gropius}, which is  based on the prototype mentioned above.
Among other things, the paper presents the architecture of the \gls{Gropius} system. 
One relevant fact for this thesis is, that the \gls{Gropius} back-end provides a \gls{graphql} \gls{API} for the front-end,
which can also be used by the tool of this thesis.

\gls{graphql} is a query language developed by Facebook, where the client needs to specify the parts of data to return \cite{grpahql2018}.
Furthermore, changed data can not just be sent to the server to be updated, but the changes must be recorded and split into so-called mutations,
which can then be sent to the server.

\subsection{Extending \acrlongpl{IDE}}
\label{ssec:ch2:ss1.3}
An \gls{IDE} is a software tool for programming, which typically has a rich text editor at its core 
and contains various other tools for use during software development.
The concept of a dedicated software \gls{IDE} can be traced back to the Maestro I system, first called ``Programm-Entwicklungs-Terminal-System'' (program development terminal system), short PET, developed by Softlab Munich reported about in \cite{Computerwoche1975Ide}.
Well known \glspl{IDE} today include Visual Studio \footnote{\url{https://visualstudio.microsoft.com}}, \gls{Eclipse} \footnote{\url{https://www.eclipse.org/ide/}}, NetBeans \footnote{\url{https://netbeans.org}}, and IntelliJ IDEA \footnote{\url{https://www.jetbrains.com/idea/}}.

Various \glspl{IDE} follow the general trend to allow extending a piece of software with the help of plugins \cite{chang2008issues}.
This enables developers to add custom features or integrated support for previously unsupported external tools.
The concept of integrating many tools into one environment was already investigated in 1990 by \cite{wasserman1990tool}.

One of those is \gls{Eclipse}, originally being developed by IBM using \gls{java} as proprietary software.
It was open-sourced in 2001 and is maintained by the Eclipse Foundation 
\footnote{\url{https://www.eclipse.org/org}} today \cite{burnette2005eclipse}.
Originally, it was intended as a \gls{java} \gls{IDE} but with the help of plugins support for many programming languages can be added today.
According to various online rankings \footnote{\url{https://pypl.github.io/IDE.html}}
\footnote{\url{https://www.positronx.io/top-10-best-ide-for-software-development/}},
\gls{Eclipse} is one of the most popular \glspl{IDE} at the time of writing.

Eclipse is designed to be extended by plugins. This is regularly done by the industry as well as by researchers 
because it is a simple way to create a large and powerful tool containing custom features 
without requiring the effort to build a complete tool from scratch. 
Recent papers using Eclipse as the base of their tool are for example \cite{segura2019extremo} or \cite{hajji2019onto2db}. 
The process of creating an Eclipse plugin is explained in great detail by \cite{clayberg2006eclipse}.

\subsection{The \acrlong{EMF} and \gls{Parsley} }
\label{ssec:ch2:ss1.4}
One alternative for manually implementing every part of a program is \gls{MDSD}.
With this approach, most of the software is modeled using some modeling language or tool and then generated \cite{volter2013model}. 

The \gls{EMF} is a framework for \gls{MDSD} optimized for use with \gls{Eclipse}.
At its core is the Ecore meta-model, with which models are represented \cite{steinberg2008emf}.
Such an Ecore model can for example be obtained by converting an existing \gls{UML} model.
From that, the complete source code for the data structure represented by the model can be generated.
The generation of other components such as editors is also supported but not used in this thesis.

\Gls{Parsley} is a framework on top of \gls{EMF} for creating \glspl{UI}.
It allows developers to build an \gls{Eclipse} \gls{UI} based on a data model generated by \gls{EMF} with very little setup \cite{bettini2014developing}.
Developers can use existing \gls{UI} elements provided by \gls{Parsley} and customize them to their needs using injected aspects  \cite{bettini2014developing}.
One core feature is the integrated \gls{DSL} allowing developers to specify the desired \gls{UI} in a simple language \cite{bettini2014developing}.

\section{Related Work}
\label{sec:ch2:s2}
First, the methodology used for finding the related work is presented in \cref{ssec:ch2:ss2.1}.
Then the related academic work found is introduced in \cref{ssec:ch2:ss2.2}.
Finally, \cref{ssec:ch2:ss2.3} covers existing issue management plugins for eclipse.

\subsection{Literature Research Methodology} 
\label{ssec:ch2:ss2.1}
The following procedure was used for finding related work:
First, an initial search was performed using the search terms below and Google Scholar \footnote{\url{https://scholar.google.com}} as the search engine.
Of each query, the top ten results were taken and examined.
First, results were rejected based on their title and the short excerpt Google Scholar includes in the search results page, 
then based on the abstract.
All remaining results were examined in depth to check whether they can be considered related work.
Finally, the same process was repeated with all references to English papers of any work found, 
which was classified as related work until no more such papers were identified.

The build the search terms all possible combinations of the following two parts were used.
For part number one the following terms were used:
\begin{itemize}
	\item IDE
	\item Integrated development environment
	\item Eclipse
\end{itemize}

For the second part these terms were used:

\begin{itemize}
	\item issue management
	\item issues
	\item issue trackers
	\item bug tracker
	\item defect tracker
\end{itemize}

This makes a total of 15 search terms resulting in 150 results, only considering the first ten results.
After sorting by title and excerpt a total of nine unique results remained, three after reading the abstract, and two after further analysis.
Of these two papers, the first had 12 references, of which all were rejected based on the title and the second had 18 references in the relevant section, of which also none could be identified as relevant for this work.

Additionally, the eclipse marketplace \footnote{\url{https://marketplace.eclipse.org}} was searched using the terms ``issue management'', ``defect'', and ``defects''. 
All results were searched for appropriate plugins.
The term ``bug'' could not be used to search the marketplace, because over 200 plugins were found and on the first few pages, most results were not relevant.

\subsection{\gls{IDE} Support for Issue Management in Academia} 
\label{ssec:ch2:ss2.2}
Only two papers about the integration of issue management into an \gls{IDE} could be found.
This indicates, that not a lot of research has been done in this area.

The first of the two papers is \cite{iqbal2009integrating},
which includes a short section on integrating their custom framework called ``Linked Data Driven Software Development (LD2SD)'' 
into the \gls{Eclipse} \gls{IDE}.
LD2SD allows searching in various software development artifacts like documentation or issues.
However, the only integration into \gls{Eclipse} presented in \cite{iqbal2009integrating} is the ability to perform a query based on 
parts of the source code (like a class or method) within \gls{Eclipse} and the result would be shown in the integrated web browser provided by \gls{Eclipse}.
They show that such a query can be started by a custom element in the context menu of \gls{java} class files in the \gls{Eclipse} package explorer.

Similarly to theirs, the goal of this thesis is to integrate an existing framework into the \gls{Eclipse} \gls{IDE} with the help of a plugin.
However, in contrast to \cite{iqbal2009integrating}, this work only focuses on issues but aims to provide much more extensive support to work with those, like adding custom \gls{UI} elements to view and edit them.

The second related work found is \cite{janak2009issue}.
It first gives an overview of existing issue management plugins for \glspl{IDE} and then describes the process for developing an ``Atlassian NetBeans Connector''.

In the overview, the author differentiates between two kinds of plugins realizing the integration of issue management into \glspl{IDE}.
According to \cite{janak2009issue} the first kind is a ``Universal plug-in with bridge for connecting the issue tracking system'',
which allows the easy integration of multiple \glspl{IMS} by providing an interface, which can be implemented by so-called bridges, 
each attaching a single \gls{IMS} to the universal plugin.
The core of this universal plugin is then responsible to work with the \gls{IDE} to show the retrieved issues to the user.
\Cite{janak2009issue} states, that this kind of issue management plugin generally does not support advanced features of the individual \glspl{IMS}.
The second kind is the ``Single issue tracking system plug-in'' \cite{janak2009issue}, which typically supports most features of a single \gls{IMS}.

For the first kind, he describes Mylyn for \gls{Eclipse}, which is discussed in detail in \cref{ssec:ch2:ss2.3} as well as Cube°n for Netbeans.
Cube°n \footnote{\url{https://code.google.com/archive/p/cubeon/}} is a task-focused \gls{UI} for NetBeans similar to Mylyn, which was originally started at Google's ``Summer of Code`` in 2008 \cite{janak2009issue}.
It has since been renamed to Task Focused NetBeans \footnote{\url{http://wiki.netbeans.org/TaskFocusedNetBeans}} and is being maintained by the NetBeans team and community.

For the second kind \cite{janak2009issue} introduces two plugins for the \gls{IMS} CodeBeamer ALM by Intland Software \footnote{\url{https://codebeamer.com/}}, one for \gls{Eclipse} and one for NetBeans, as well as the ``Atlassian IDE Connector'' for IntelliJ IDEA and \gls{Eclipse}.
According to the author, the CodeBeamer ALM is a good issue management plugin supporting all CodeBeamer ALM features with a simple and intuitive \gls{UI}.
According to \cite{janak2009issue} both Atlassian Connectors allow the use of various Atlassian products, including Jira, in the \gls{IDE}.
The author states that the connector for IntelliJ IDEA is a lot more advanced than the one for \gls{Eclipse}, which is based on Mylyn.
The Atlassian Connector for \gls{Eclipse} is also covered in \cref{ssec:ch2:ss2.3}, but in \cite{janak2009issue} it is explained.

Then \cite{janak2009issue} describes the concept, design, and implementation of a new Atlassian connector for NetBeans, with the Jira connector being the primary goal.
According to the author, it is primarily inspired by the above mentioned Atlassian IntelliJ IDEA connector.
The created Jira integration allows users to view a filtered list of issues, a specific issue as well as their comments, create a new issue or comment, log work on an issue, view stack traces from an issue, and click through to the relevant source file, assign an issue to someone and perform workflow actions (similar to changing the state) on a selected issue.

That work provides a much more advanced integration than the previous one, with a similar amount of features to the one presented in this thesis.
On the other hand, the plugin presented in \cite{janak2009issue} only works with the Jira \gls{IMS} in contrast to the one of this thesis which integrates into the \gls{Gropius} framework, which theoretically can work with any \gls{IMS}.
Additionally, theirs is for NetBeans, while ours is for \gls{Eclipse}.

However, the most important difference between this thesis and the two covered related works
is the fact that, in contrast to the other two, the tool presented in this thesis has support for the features of \gls{Gropius},
which are intended to help with the issue management in multi-component, multi-team environments.


\subsection{Issue Management Plugins for Eclipse}
\label{ssec:ch2:ss2.3}
Compared to academia, the industry has put a little more work into this area.
There exist a few issue management plugins for various \glspl{IDE}, as also shown in \cite{janak2009issue}, 
but in this section, the focus is on issue management plugins for \gls{Eclipse}.
In total five issue management plugins for \gls{Eclipse} could be found.
Four, which are specific to one \gls{IMS} and one universal plugin using bridges to connect to many \glspl{IMS}.

The first is ``Teamscale Integration for Eclipse'' \footnote{\url{https://marketplace.eclipse.org/content/teamscale-integration-eclipse}}, 
which allows users to browse the defects found by a Teamscale software‐quality analysis server.
This server is not a typical \gls{IMS}, but rather a software analyzing the source code for potential issues.

The second is called ``CollabNet Desktop'' \footnote{\url{https://marketplace.eclipse.org/content/collabnet-desktop-eclipse-edition}} and 
integrates with the application lifecycle management platforms of CollabNet, which include issue management and can therefore be called \gls{IMS} in the context of this thesis.

The next is ``JiraBuddy - Eclipse Plugin for JIRA'' \footnote{\url{https://marketplace.eclipse.org/content/jirabuddy-eclipse-plugin-jira}}
is a plugin adding various enhancements when working with Atlassian Jira and \gls{Eclipse}, but does not provide the capability to create or update 
issues from within \gls{Eclipse}. According to their website \footnote{\url{http://home.jirabuddy.com/}}, the only feature is to display the issue when hovering over the corresponding issue identifier in the \gls{Eclipse} editor.

The last plugin of this kind is ``Atlassian Connector for Eclipse'' \footnote{\url{https://marketplace.eclipse.org/content/atlassian-connector-eclipse}},
which not only provides integration into Atlassian Jira, which is relevant for this thesis but also other Atlassian products like the Bamboo continues integration and deployment server.
It is actually built on top of Mylyn, which is explained below.

All of the above only work with one specific \gls{IMS}, whereas the tool proposed by this thesis can work with any \gls{IMS} supported by \gls{Gropius}.
Additionally, the last two of them are not maintained anymore and therefore have no support for current eclipse versions. 

The fifth and most important plugin is Mylyn \footnote{\url{https://marketplace.eclipse.org/content/mylyn}} together with its many connectors for various \glspl{IMS}.
It is one of the oldest issue management plugins for any \gls{IDE} \cite{janak2009issue}.
At its core is a list of tasks, which can contain local tasks stored on the computer and tasks provided by one of the many connectors.
One important concept is the task-focused interface, which means Mylyn tries to highlight the parts of the \gls{Eclipse} \gls{UI} relevant to the current task and makes the less relevant parts less prominent.
That mainly applies to the \gls{Eclipse} package explorer, where different resources and files will be drawn with different shades of gray based on their relevance for the current task.
Data about the relevancy of parts for the current task is extracted from the user's behavior.
But in contrast to the solution provided in this thesis, Mylyn can only associate issues, or tasks as they are called in Mylyn, to resources as specific as source code files and not to lines within them.

Similar to the academic work, all these plugins lack support for all or some features necessary in multi-component, multi-team environments,
like one issue being saved in multiple different \glspl{IMS} of multiple teams or the ability to quickly navigate between related issues.


% !TeX spellcheck = en_US

\chapter{Concept}
\label{chap:ch3}
The goal of this work is to evaluate the usefulness of an \gls{IDE} integration for the Gropius system \cite{speth2020gropius}.
This chapter describes the approach taken in this thesis for archiving that goal.
At the heart of the concept is the plan to develop a prototype issue management plugin for one \gls{IDE},
which integrates into the Gropius system.
This tool can then be used in the evaluation by presenting to experts in the field for a review.

The first step was to gather requirements for that prototype. 
This requirements engineering process as well as its results are described in \cref{sec:ch3:s1}.
As visualizing details about an issue in the corresponding icon is a key part of multiple requirements,
the next step was to determine which properties to visualize and how.
The resulting concept for the icons is explained in \cref{sec:ch3:s2}.
Finally, as described in \cref{sec:ch3:s3}, 
a rough concept of the prototype tool was made before continuing with the design and implementation.

\section{Analysis and Requirements Engineering}
\label{sec:ch3:s1}
To gather requirements for the plugin prototype a process with six steps as shown in \cref{fig:requirmentsProcess} was used.
This process can be split into two stages, each with the three activities of requirements elicitation, specification, and validation.

\begin{figure}[!h]
	\centering
	\tikzstyle{mybox} = [draw=blue!40, fill=blue!20, very thick,
	rectangle, rounded corners, inner sep=5pt, inner ysep=5pt,
	minimum width=0.45\linewidth, minimum height=80pt, text width=0.45\linewidth]
	\tikzstyle{title} = [fill=blue!40, rectangle, rounded corners]
	\begin{tikzpicture}
	\node[mybox] (box1) {
		\begin{varwidth}{\linewidth}\begin{itemize}
		\item Internal Brainstorming
		\end{itemize}\end{varwidth}
	};
	\node[title] (box1title) at (box1.north) {Requirements Elicitation};
	
	\node[mybox] (box2) [right=of box1] {
		\begin{varwidth}{\linewidth}\begin{itemize}
		\item Artifact: Thesis Proposal Paper
		\end{itemize}\end{varwidth}
	};
	\node[title] (box2title) at (box2.north) {Requirements Specification};
	
	\node[mybox] (box3) [below=of box2] {
		\vskip 0.3\baselineskip
		\begin{varwidth}{\linewidth}\begin{itemize}
		\item Thesis Proposal Paper Feedback
		\item Thesis Proposal Presentation Discussion and Feedback
		\end{itemize}\end{varwidth}
	};
	\node[title] (box3title) at (box3.north) {Requirements Validation};
	
	\node[mybox] (box4) [below=of box1] {
		\vskip 0.3\baselineskip
		\begin{varwidth}{\linewidth}\begin{itemize}
		\item Feedback from Thesis Propsal
		\item Internal Brainstorming
		\item Stakeholder Interview
		\end{itemize}\end{varwidth}
	};
	\node[title] (box4title) at (box4.north) {Requirements Elicitation};
	
	\node[mybox] (box5) [below=of box4] {
		\vskip 0.3\baselineskip
		\begin{varwidth}{\linewidth}\begin{itemize}
		\item Artifact: Requirements Document
		\end{itemize}\end{varwidth}
	};
	\node[title] (box5title) at (box5.north) {Requirements Specification};
	
	\node[mybox] (box6) [below=of box3] {
		\vskip 0.3\baselineskip
		\begin{varwidth}{\linewidth}\begin{itemize}
		\item Review through Supervisor
		\end{itemize}\end{varwidth}
	};
	\node[title] (box6title) at (box6.north) {Requirements Validation};
	
	\path [->,draw,thick] (box1) -- (box2);
	\path [->,draw,thick] (box2) -- (box3title);
	\path [->,draw,thick] (box3) -- (box4);
	\path [->,draw,thick] (box1) -- (box4title);
	\path [->,draw,thick] (box4) -- (box5title);
	\path [->,draw,thick] (box5) -- (box6);
	\end{tikzpicture}
	\caption{Requirements Engineering process}
	\label{fig:requirmentsProcess}
\end{figure}

As the first step of the first stage, a preliminary requirements elicitation was performed. 
For this, internal brainstorming and discussion with the thesis' supervisors were used.
Next, the resulting requirements were incorporated into the paper for the thesis proposal, 
which serves as the first requirements specification.
After this, the primary requirements validation was performed. 
The positive feedback for that proposal paper was one of the signs that the requirements are valid.
Additionally, the discussion and feedback after the thesis proposal presentation was also used to validate the requirements.

Then the second stage of requirements elicitation was performed using the requirements from the first elicitation, 
the results from the validation as well as the results of a stakeholder interview.
For this interview, the general concept of Gropius as well as of this thesis was sent to representatives of each group of stakeholders. 
They were then asked for their requirements with the help of two questions.
The first question asked for features expected from an issue management plugin in general,
the second for those, which the representatives would expect from such a plugin, which integrates into Gropius.

As the prototype is a new tool for an unfinished system, only three groups of stakeholders could be identified:
First, the potential users, in this case developers, working with multi-component systems.
Various members of the department for software quality and architecture were contacted as representatives of this group.
Second, the thesis' supervisors to ensure the scientific scope of the work. 
Third, Sandro Speth as the author of the Gropius system \cite{speth2020gropius}.

Next, the gathered requirements were formally specified as user stories in the requirements document.
Finally, this document was validated by a review from the thesis' supervisors.


\subsection*{Gathered Requirements}
The result of this requirements engineering process is a list of functional requirements in the form of user stories.
These requirements are listed below, split into those expected from any issue management plugin and those expected from a plugin integrated into Gropius.

\newcounter{enumarteCounter} % For saving the counter between the two enumerates

First, the requirements expected from any issue management plugin:
\begin{enumerate}
	\item As a developer I want to have a list of all open issues. \label{itm:ch3:req:filter_open}
	\item As a developer I want to have a list of all issues assigned to me. \label{itm:ch3:req:filter_me}
	\item As a developer I want to have a searchable list of issues. \label{itm:ch3:req:filter_search}
	\item As a developer I want to have a list of issues that can be filtered by tags or labels. \label{itm:ch3:req:filter_labels}
	\item As a developer I want to have a list of issues which can be filtered to only show issues relevant to my currently opened project. \label{itm:ch3:req:filter_open project}
	\item As a developer I want to have a list of issues which can be filtered to only show issues relevant to my currently opened source file. \label{itm:ch3:req:filter_open_file}
	\item As a developer I want to quickly see the type of an issue in this list. \label{itm:ch3:req:list_issue_type}
	\item As a developer I want to be able to see all details for a specific issue.
	\item As a developer I want to be able to jump to a related issue from my currently viewed issue.
	\item As a developer I want to be able to quickly open a relevant source file at the relevant line for an issue, even if it is in another eclipse project (as long as that project is in my workspace).
	\item As a developer I want to be shown which lines of a source file are relevant for an issue. (For example through markers at the side of the editor) \label{itm:ch3:req:source_file_marker}
	\item As a developer I want to be able to edit every property of an issue that I have write access to.
	\item As a developer I want to be able to create a new issue.
	\item As a developer I want to be able to create a new issue for some lines of code without manually entering the source file and lines in a dialog.
	\setcounter{enumarteCounter}{\value{enumi}} % store counter to continue later
\end{enumerate}

These are the requirements expected from such a plugin integrated into Gropius:
\begin{enumerate}
	\setcounter{enumi}{\value{enumarteCounter}} % restore counter from above
	\item As a developer I want the system architecture graph in my web browser to focus on the component I've selected in eclipse.
	\item As a developer I want the system architecture graph in my web browser to highlight the issue I've selected in eclipse.
	\item As a developer I want to be able to open details about an issue in the eclipse view through the architecture graph in my web browser.
	\item As a developer I want to be able to open a relevant source code line of an issue in eclipse from the architecture graph in my web browser.
	\item As a developer I want to be able to have the possibility to set the focus of the architecture graph and eclipse editor as well as the issue view of my colleagues, for example in meetings.
\end{enumerate}

\section{Issue Icons}
\label{sec:ch3:s2}
For requirements \ref{itm:ch3:req:list_issue_type} and \ref{itm:ch3:req:source_file_marker}, it is useful to have icons for each issue.
Especially for requirement \ref{itm:ch3:req:list_issue_type} the icons can not all be the same but need to carry some information.
Therefore, a concept was made about which properties should be visualized in these icons.

As it was not clear how much information could be visualized in one small icon, 
the properties that should be visualized were ordered by their priority.
That way it was easier to make decisions about what information to include in the item and how prominent to visualize it later during the creation of these icons.
The following properties were chosen to be visualized with decreasing priority.
\begin{itemize}
	\item Type of Issue
	\subitem Enhancement vs. Bug
	\item Does this issue cause problems in other projects?
	\item Is this issue just a symptom of another issue elsewhere?
	\item Is the issue assigned to the viewing developer?
	\item Is the issue new or work in progress or done?
\end{itemize}
Finally, another aspect, that could be visualized is the visibility of the issue as some \glspl{IMS} can have issues that are only visible internally.
But it was decided against it for now, as it didn't seem that important of a property and the Gropius system currently doesn't support it anyway.

\section{Prototype Plugin}
\label{sec:ch3:s3}
Before starting with the detailed design and implementation of the prototype tool, a rough concept for it needed to be created.
Most parts of this were already done during the thesis' proposal process and a version of this concept was included in the proposal paper.
Therefore, the feedback for the paper and the presentation in front of the department was a confirmation, that the general concept is reasonable.

According to the concept, the plugin consists of two main \gls{UI} elements: The issue list and the issue details element.
These are complemented by various dialogs and other features described below.
The plugin uses the \gls{API} of the Gropius system \cite{speth2020gropius} to retrieve its data and perform any modifications on it.
Furthermore, the plugin is split into multiple components, 
of which only one is specific to the supported \gls{IDE} 
and the others are reusable for other \glspl{IDE} with minimal changes.
That way the amount of work to port it for another \gls{IDE} is minimized,
as long as it supports java-based plugins.

\begin{figure}[!h]
	\centering
	\includegraphics[width=\textwidth]{graphics/concept_mockup_issueList.png}
	\caption{Mock-up for Issue List}
	\label{fig:c3:mockup_issueList}
\end{figure}
The issue list is a table of issues with columns for the various properties of those, 
similar to the one shown in the mock-up in \cref{fig:c3:mockup_issueList}.
In the first column, the icon for the issue is shown followed by the title.
This table allows filtering by various attributes of the issues.
To satisfy requirements \ref{itm:ch3:req:filter_open} to \ref{itm:ch3:req:filter_open_file} it supports at least filtering
by issue state, assignee, text contained in the title or description, 
labels as well as whether the issue is relevant to the open \gls{IDE} project or the open file.
Furthermore, it is possible to sort the table based on any column. 
Finally, the visible columns can also be configured by the user.

\begin{figure}[!h]
	\centering
	\includegraphics[width=0.5\textwidth]{graphics/concept_mockup_issueDetails.png}
	\caption{Mock-up for Issue Details}
	\label{fig:c3:mockup_issueDetails}
\end{figure}
The issue details \gls{UI} element shows all attributes for the issue selected in the issue list.
A mock-up of it can be seen in \cref{fig:c3:mockup_issueDetails}.
At the top of the form, the icon of the issue is displayed.
The related issues are links, which, when clicked, change the issue list selection to the corresponding issue.
Furthermore, the displayed locations are also links, which open the correct resource in the \gls{IDE}'s editor.
Additionally, this element supports editing the issue, either by opening a dialog or by allowing to edit the values in the element itself like a form.
In both cases, additional dialogs are used to allow editing of the more complex attributes. 

\begin{figure}[!h]
	\centering
	\includegraphics[width=0.75\textwidth]{graphics/concept_mockup_issueMarker.png}
	\caption{Mock-up for Issue Markers}
	\label{fig:c3:mockup_issueMarkers}
\end{figure}
Moreover, the issues are displayed as markers in the \gls{IDE}'s editor, whenever a file with issues is open in the editor.
This can be seen in the mock-up in \cref{fig:c3:mockup_issueMarkers}. 
For these the icons from \cref{sec:ch3:s2} are used, if the \gls{IDE} supports it.

Additionally, there are two ways to create a new issue.
First, there is a button somewhere, which creates a new empty issue, 
optionally opening a dialog to enter details for the new issue.
Second, it is possible to mark some lines in the editor and create a new issue for them.
In this case, the new issue already has the correct location selected.

\begin{figure}[!h]
	\centering
	\includegraphics[width=0.5\textwidth]{graphics/concept_gropius_frontend_messaging.png}
	\caption{Messaging Concept for Gropius Frontend Integration}
	\label{fig:c3:concept_gropius_messaging}
\end{figure}

Finally, the tool is connected to the existing web \gls{UI} with the help of a messaging component.
That way it is possible to enable synchronization between the plugin and the web user interface, 
such that selecting an issue in one will cause the other to also jump to the same issue and select it accordingly. 
Each user has a separate topic in the messaging component, preventing undesirable interaction between multiple users. 
In addition to this, special topics for example for meetings are a possibility, 
such that one developer can select an issue and the same is selected for all other participants of the meeting. 
\cref{fig:c3:concept_gropius_messaging} shows the basic architecture planned for this feature. 
The plugin, as well as the web user interface, communicate with the back-end through a \gls{REST} \gls{API} and with each other using a messaging component.

% !TeX spellcheck = en_US

\chapter{Design and Implementation}
\label{chap:ch4}


% !TeX spellcheck = en_US

\chapter{Evaluation}
\label{chap:ch5}
For evaluating the usefulness of the prototype an expert review and survey is performed.
To acquire the questions for this survey a process based on the \gls{GQM} approach introduced by \cite{caldiera1994goal} is used.
The results from that are shown in \cref{sec:ch5:s1}.
\Cref{sec:ch5:s2} describes the design and realization as well as the results of the expert review and survey.
In \cref{sec:ch5:s3} the results of the survey as well as the validity of the thesis are discussed.
Finally, the identified threats to this validity are described in \cref{sec:ch5:s4}.

\section{GQM}
\label{sec:ch5:s1}
The process described in this section does not exactly follow the \gls{GQM} approach introduced by \cite{caldiera1994goal},
but is based on it.
The objective of the process is to find good questions, which can be used to evaluate the success of this thesis.
Based on these questions the expert survey can then be performed.

The first step, is to identify the goals of the work.
For this thesis the following could be identified as the primary and only goal \textbf{(G1)}:
Improve the efficiency, effectiveness, and convenience for developers when working with issues in the Gropius environment during their software development process. 

Next, questions have to be found, which can be used to evaluate, whether that goal is reached.
For that purpose the following three questions could be identified:
\begin{itemize}
	\item[\textbf{Q1}] What problems do developers face when working with issues affecting multiple independent projects during the software development/engineering process?
	\item[\textbf{Q2}] Which of these problems does this thesis's tool solve and to which degree?
	\item[\textbf{Q3}] Can the industry imagine using such a tool in production?
\end{itemize}

As it was clear from the start, that an expert review and survey would be conducted, no metrics need to be found for these questions.
However, it must be investigated if an expert review is a reasonable metric for these questions.
For \textbf{Q1} an expert survey is a reasonable metric, if the experts are developers, who currently work are have worked in a field with multiple independent projects or have enough experience to put them self into that situation.
It is an adequate metric for \textbf{Q2}, if the experts fulfill the conditions above and have used the thesis' tool or at least were given an introduction about the features of the tool.
An expert survey is a reasonable metric for \textbf{Q3}, if the experts are or have been working in the industry and fulfill the above conditions.

Therefore, the evaluation metric (\textbf{M1}) for all questions is an expert review and survey with the correct experts, who are introduced to \gls{GropiusEI}. 

\section{Expert Survey}
\label{sec:ch5:s2}
The main element of this evaluation is the expert survey, where selected experts answer a set of prepared questions. 
But first an expert review of \gls{GropiusEI} is performed.
This makes sure the experts know the tool and its features and has the additional benefit of direct feedback about the prototype.
The process of the review and survey is explained in \cref{ssec:ch5:ss2.1}.
The results from both are presented in \cref{ssec:ch5:ss2.2}.

\subsection{Design and Realization}
\label{ssec:ch5:ss2.1}
To perform an expert review and survey, first a group of experts has to be found.
To achieve this various developers from academia and industry were contacted and asked to participate in this review.
If they were generally interested, the thesis' concept and the procedure of the review and survey were explained to them.
In total 21 experts where contacted, of which 12 agreed to take part.

As preparation for the expert review, a detailed guide on installing \gls{GropiusEI} was produced
to ensure experts could install it on their own device if they wanted.
It can be found in \cref{sec:appendix:er:install_guide} and
the exact release used for the review is attached as \cref{sec:appendix:er:release}.

Additionally, a scenario was created, in which experts could test the tool in as real as possible conditions.
This scenario consists of an eclipse workspace with a java-based demo software using multiple components \footnote{\url{https://github.com/eventuate-tram/eventuate-tram-examples-customers-and-orders}} and a \gls{GropiusEI} data file.
This data file contains the correct components and interfaces for the workspace as well as some developers and labels in addition to
a number of invented issues.
These issues are as realistic as possible with a title that could be found in a live \gls{IMS} somewhere, a feasible text body,
typical labels, sometimes linked issues, and most with at least one location in the workspace.
Furthermore, a few tasks were prepared, which were used to introduce each feature to the expert.
These tasks were typical things a developer would need to do during his workday like finding an issue in the code are creating a new issue.
For each task a detailed explanation, how the task can be done using \gls{GropiusEI} is also included. 
The scenario can be found in \cref{sec:appendix:er:scenario} and a short explanation as well as the mentioned tasks in \cref{sec:appendix:er:scenario_explanation}.

After an expert agreed to participate they were provided with three options for executing the expert review and asked which they preferred.
The first option was to install the tool and the scenario on their own device with the help of the mentioned installation guide.
Then an online meeting would be scheduled, such that the tasks from the scenario could be completed while I watch with the help of a screen sharing solution. That way I could help the experts with any problems and directly answer questions as well as guide them through the prepared tasks instead of sending the tasks.
The second option was, that the expert installs both on their own device and go through the scenario doing the tasks on their own.
In this case, they were sent the tasks including the explanations in a written document.
The last option was, that I would present the tool in an online meeting using screen sharing tools. That way the expert did not need to install anything on their device. For that reason, this was also the fastest option. On the other hand, the expert could not use the tool himself, but just watch me do the tasks.
In every case, the expert was asked for general feedback.

After an expert completed the review he was sent the following questions and asked to answer them:
\begin{enumerate}
	\item What do you see as the most significant problems when working with issues affecting multiple independent projects during your software development/engineering process? \label{itm:ch5:expert_questions_problems}
	\item Do you think the following points represent problems or challenges of existing issue management systems you face as a developer? \label{itm:ch5:expert_questions_are_these_problems}
	\subitem The lack or insufficient support of linking issues to multiple specific software development artefacts accurate to the line, especially if the artifacts are part of multiple independent projects.
	\subitem The lack or insufficient support of linking issues to each other
	\subitem The lack or inadequate support of navigating from one to the next issue through the chain of related or dependent issues.
	\item For which of these problems (from points 1 and 2) would you say Gropius EI completely solves them? \label{itm:ch5:expert_questions_solve}
	\item For which of these problems (from points 1 and 2) would you say Gropius EI reduces them? \label{itm:ch5:expert_questions_reduce}
	\subitem How far would you say are each of these problems reduced?
	\item Could you imagine to use Gropius EI in your day-to-day business?
\end{enumerate}
These questions are based on the evaluation questions \textbf{Q1} to \textbf{Q3} presented in \cref{sec:ch5:s1}.
Question \ref{itm:ch5:expert_questions_are_these_problems} was added in order to get good answers to questions \ref{itm:ch5:expert_questions_solve} and \ref{itm:ch5:expert_questions_reduce} even if the expert did not find a lot of answers to question \ref{itm:ch5:expert_questions_problems}.
The points for question \ref{itm:ch5:expert_questions_are_these_problems} were conceived in an internal brainstorming session.

\subsection{Results}
\label{ssec:ch5:ss2.2}
%Of the 12 experts, who agreed to participate, a total of $x$ reviewed the tool and $y$ answered the prepared questions in the end. \todo{Insert correct numbers}
\section{Discussion}
\label{sec:ch5:s3}
\section{Threats to Validity}
\label{sec:ch5:s4}

\begin{figure}[!h]
	\centering
	\tikzset{
		my node style/.style={
			font=\small,
			rectangle,
			rounded corners,
			minimum size=6mm,
			drop shadow,
			draw=blue!40, 
			fill=blue!20, 
			very thick,
			align=center,
		}
	}
	\forestset{
		my tree style/.style={
			for tree={
				my node style,
				parent anchor=south,
				child anchor=north,
				l sep+=5pt,
				edge={draw=blue!40, very thick},
				edge path={
					\noexpand\path [draw, \forestoption{edge}] (!u.parent anchor) -- +(0,-7.5pt) -| (.child anchor)\forestoption{edge label};
				},
			    delay={if content={}{shape=coordinate}{}}
			}
		}
	}
	\begin{forest}
	my tree style
	[Threats to Validity
      [External Validity
	    [\begin{varwidth}{0.16\linewidth}Small sample size\end{varwidth}]
	    [\begin{varwidth}{0.16\linewidth}Sample not representative\end{varwidth}]
	  ]
	  [Construct Validity 
	    [[[
	      [\begin{varwidth}{0.16\linewidth}Questions for experts misrepresent research questions\end{varwidth}]
	      [\begin{varwidth}{0.16\linewidth}Questions misunderstood by experts\end{varwidth}]
	      [\begin{varwidth}{0.16\linewidth}Answers misinterpreted\end{varwidth}]
	      [\begin{varwidth}{0.16\linewidth}Only one kind of measurement\end{varwidth}]
	      [\begin{varwidth}{0.16\linewidth}Expert survey is not objective nor statistical\end{varwidth}]
	    ]]]
	  ]
	  [Reliability
	      [\begin{varwidth}{0.16\linewidth}Documents written in German\end{varwidth}]
	      [\begin{varwidth}{0.16\linewidth}Artifacts hosted online may not be available forever\end{varwidth}]
	  ]
	  [Internal Validity]
	]
	\end{forest}
  
	\caption{Overview over the Threats to Validity}
	\label{fig:threatsToValididty}
\end{figure}

This section discusses which threats to the validity of the results discussed above could be identified.
An overview of these threats can be seen in ...
The threats can be grouped into four categories as stated by \cite{runeson2009guidelines}.

The first category investigated, is the internal validity.
This aspect of validity is concerned with whether the observed results are actually caused by the changes introduced by this work,
or if the results could be effects of any unknown influence \cite{runeson2009guidelines}.
But as the experts were directly asked if \gls{GropiusEI} solves or reduces the stated problems, no threat to this aspect of validity could be identified.

The next kind is the construct validity.
It represents the correctness of the assumption, that the collected results actually correspond to the research questions \cite{runeson2009guidelines}.
For this aspect a few threats could be identified.
First, the questions sent to the experts, which are presented in \cref{ssec:ch5:ss2.1} could misrepresent the research questions 
introduced in \cref{sec:ch5:s1}.
Next, these questions could have been misunderstood by one or more experts.
Furthermore, the answers of the experts could have been misinterpreted.
Moreover, just a single kind of measurements was performed.
Finally, an expert survey does not produce objective ore statistical results.
The results are subjective answers from the experts given in free text form, such that statistical evaluation is difficult. 

Another aspect is the external validity.
That is concerned with whether the results can be generalized beyond the small studied sample \cite{runeson2009guidelines}.
The following threats were identified for this category:
First, the sample size was very small compared to all developers working with issues in a relevant field.
Second, the experts were not picked randomly, but all had prior contact to me or my supervisor.
Therefore, the sample was not very representative of the complete population.

The last aspect of validity is the reproducibility or reliability.
It deals with any reasons a separate evaluation by other researchers reproducing this evaluation would not have the same conditions \cite{runeson2009guidelines}.
For this aspect no major threats could be identified, as the whole process is described in detail in \cref{ssec:ch5:ss2.1} and all artifacts used can be found in \cref{chap:appendix:expert_review_docs}.
The first minor threat found is the fact that these documents are written in German and would need to be translated for researchers or experts who don't speak German.
The other minor threat is that the artifacts hosted on the internet referenced in the appendix might not be available forever.
% !TeX spellcheck = en_US

\chapter{Conclusion}
\label{chap:chap6}

\section*{Future Work}

\printbibliography

All links were last followed on August 17, 2020.

\appendix
%\input{latexhints-english}

\pagestyle{empty}
\renewcommand*{\chapterpagestyle}{empty}
\Versicherung
\end{document}
